\documentclass{article}
\usepackage{graphicx} % Required for inserting images

% Layout packages
\usepackage[margin=1in]{geometry}
\usepackage{multicol}

% Formatting packages
\usepackage{parskip}

% Bibliography packages
\usepackage[
backend=biber,
style=numeric,
sorting=none
]{biblatex}
\addbibresource{PhD Summer Project.bib}

% Body of document
\title{Summer Project Report}
\author{Andrew Poile}
\date{August 2025}

\begin{document}

\maketitle

\begin{multicols}{2}
\section{Introduction}
\subsection{School allocation}
In the 1988 Education Reform Act, the UK introduced school choice mechanisms as a way to leverage the power of market mechanisms, which would introduce an element of competition between schools and incentivise them to improve. In the United Kingdom, and many places besides, primary and secondary school students are allocated to schools using a centralised mechanism, according to preference lists generated by the students (or rather their parents).

In the United Kingdom the acceptance of students also depends on the bespoke priority lists which exist to determine acceptance in the case of over-subscription to the school, that is to say, if a school has more applicants than seats. The local authority (LA) (usually the local city council) is the admissions authority for community schools and voluntary-controlled schools. Some schools can be their own admissions authorities, for example academies, voluntary aided schools (usually religious), foundation schools, and trust schools can all be their own admissions authorities and can set their own over-subscription criteria. Generally, the difference between the two is the degree of funding that they receive from the local authority, and the trade-off is autonomy in admissions.

Priority lists are bespoke, but all maintained schools are beholden to the School Admissions Code, which is a statutory document issued by the government, that imposes mandatory requirements and guidelines; for example, all local authorities must prioritise children in care above all other students. Some over-subscription criteria are common over almost all schools, namely the prioritisation of students who have a sibling in the school, and the prioritisation of students who live geographically closest to the school.

Parents must be allowed at least three preferences for secondary schools but London borough schools allow up to six. 

Fee-paying independent schools (private schools) are altogether exempt from the school admissions code and have complete freedom in their pupil selection.

\section{Literature review}
\subsection{Allocation mechanisms}
The allocation mechanism describes the algorithm by which the resources are allocated, in this case students allocated to schools. School choice mechanisms are a special variant of the deferred acceptance mechanism in which only one side expresses preferences (students) and the other has a universally known priority ranking (schools). Furthermore, fairness criteria only consider the students' allocations since schools cannot express preferences. 

In England allocation is exclusively done by variations of deferred acceptance (DA) mechanisms, with immediate acceptance (IA) mechanisms (also First-Place-First, or Boston mechanisms) being explicitly banned in 2008 \cite{pathakSchoolAdmissionsReform2013}\cite{terrierImmediateAcceptanceDeferred} due to concerns about fairness, transparency and simplicity. Random allocation as the principal mechanism for allocation is also prohibited in the statutory documents.

The additional restriction on the length of students' preference lists has a significant effect on the mechanism however; they remove the property of DA mechanisms to be dominant-strategy truthful (DST), reintroducing a level of strategy into the participation process for students \cite{haeringerConstrainedSchoolChoice2009}. 

\subsection{Socio-economic segregation in schools and residential areas}
Empirical evidence suggests that school choice has increased both school segregation (the differences between schools) and residential segregation \cite{wilsonSchoolChoiceCity2019}. Particularly in England, the geographical constraints of the system with respect to the proximity-based over-subscription criteria allow for advantaged families to achieve better outcomes by moving into the neighbourhood of their preferred school \cite{burgessParentalChoicePrimary2011}\cite{burgessSchoolChoiceEngland2019}. This restriction in choice has the effect of re-introducing the geographical inequities that the school choice system is supposed to ameliorate, since disadvantaged families who cannot afford to live in the neighbourhood of popular (and usually well-performing) schools will, in the best case be lower in the priority list, and in the worst case displaced from their homes as the housing prices in the neighbourhood rise.

The central issue is the ubiquity of the proximity-based over-subscription criterion employed by the vast majority of schools in England.

Importantly, there is an implicit assumption that parents (as consumers) will rank higher performing schools higher in their preference lists, but in reality many parents will still treat their choices strategically due to the perceived impossibility of attaining a popular, performant school. Before the 2008 ban on IA mechanisms in England, advantaged parents would strategically ration their top ranked preference so as not to waste it on a school they may not be able to attain, which left room for disadvantaged but ambitious parents to obtain a few seats, but the shift away from IA mechanisms has eliminated any risk from applying for the best school in the area, making them far more popular, and geographical distance based over-subscription priorities incentivise advantaged parents to move into an area with better schools, simultaneously pushing out less-advantaged families and raising house prices in that area. This is a vicious cycle as the less-advantaged are forced into even less-advantaged areas, and given access to poorer educational opportunities all while the value of their property drops, which further cements their position.

\subsection{Effects of segregation on urban environments}


\subsection{Effects of transportation access on segregation}

\subsection{Allocation and intervention literature}


\section{Methodology}
\subsection{Bundle allocation}


\section{Impact}
\subsection{Sustainable Development Goals}
Major goals: 4 (Quality Education), and 11 (Sustainable Cities and Education).
Minor goals: 1 (No Poverty), and 13 (Climate Action).

Increasing the number of children using micro-mobility may make them vulnerable to injury on the roads, especially in the case of e-scooters ridden without helmets.

Inducing demand for public transit will incentivise the development and improvement of infrastructure, which will lead to greater adoption of public transit, and thus a virtuous cycle will emerge that improves public transit and adoption.

\subsection{Changes}
Introducing changes to priorities and presenting schools alongside transport options (discounts?) as bundles to influence preferences may reduce dependence on private vehicle use, producing voting coalitions which demand improvements to public goods, leading to social and environmental sustainability.

\section{Future work}
\end{multicols}

% Bibliography
\break
\printbibliography

\end{document}
