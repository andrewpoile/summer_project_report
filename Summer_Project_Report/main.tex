\documentclass{article}
\usepackage{graphicx} % Required for inserting images

% Layout packages
\usepackage[margin=1in]{geometry}
\usepackage{multicol}

% Formatting packages
\usepackage{parskip}

% Bibliography packages
\usepackage[
backend=biber,
style=numeric,
sorting=none
]{biblatex}
\addbibresource{PhD Summer Project.bib}

% Body of document
\title{Summer Project Report}
\author{Andrew Poile}
\date{August 2025}

\begin{document}

\maketitle

\begin{multicols}{2}
\section{Introduction}
\subsection{School allocation}
In the 1988 Education Reform Act, the UK introduced school choice mechanisms as a way to leverage the power of market mechanisms, which would introduce an element of competition between schools and incentivise them to improve. In the United Kingdom, and many places besides, primary and secondary school students are allocated to schools using a centralised mechanism, according to preference lists generated by the students (or rather their parents).

In the United Kingdom the acceptance of students also depends on the bespoke priority lists which exist to determine acceptance in the case of over-subscription to the school, that is to say, if a school has more applicants than seats. The local authority (LA) (usually the local city council) is the admissions authority for community schools and voluntary-controlled schools. Some schools can be their own admissions authorities, for example academies, voluntary aided schools (usually religious), foundation schools, and trust schools can all be their own admissions authorities and can set their own over-subscription criteria. Generally, the difference between the two is the degree of funding that they receive from the local authority, and the trade-off is autonomy in admissions.

Priority lists are bespoke, but all maintained schools are beholden to the School Admissions Code, which is a statutory document issued by the government, that imposes mandatory requirements and guidelines; for example, all local authorities must prioritise children in care above all other students. Some over-subscription criteria are common over almost all schools, namely the prioritisation of students who have a sibling in the school, and the prioritisation of students who live geographically closest to the school.

Parents must be allowed at least three preferences for secondary schools but London borough schools allow up to six. 

Fee-paying independent schools (private schools) are altogether exempt from the school admissions code and have complete freedom in their pupil selection.

\section{Literature review}
\subsection{Allocation mechanisms}
The allocation mechanism describes the algorithm by which the resources are allocated, in this case students allocated to schools. School choice mechanisms are a special variant of the deferred acceptance mechanism in which only one side expresses preferences (students) and the other has a universally known priority ranking (schools). Furthermore, fairness criteria only consider the students' allocations since schools cannot express preferences. 

In England allocation is exclusively done by variations of deferred acceptance (DA) mechanisms, with immediate acceptance (IA) mechanisms (also First-Place-First, or Boston mechanisms) being explicitly banned in 2008 \cite{pathakSchoolAdmissionsReform2013,terrierImmediateAcceptanceDeferred} due to concerns about fairness, transparency and simplicity. Random allocation as the principal mechanism for allocation is also prohibited in the statutory documents. 

The restriction on the length of students' preference lists has a significant effect on the mechanism; it removes the property of DA mechanisms to be dominant-strategy truthful (DST), reintroducing a level of strategy into the participation process for students \cite{haeringerConstrainedSchoolChoice2009} with manipulability being inversely proportional to the number of choices offered to students \cite{pathakSchoolAdmissionsReform2013} and empirical evidence suggests that people do strategise \cite{burgessSchoolChoiceEngland2019}, with top-ranked preferences being more ambitions with access to multiple choices, but safer when restricted to fewer choices. 

The DA mechanism is the mechanism that provides the most efficient stable matching when preferences and priorities are strict. This is important because if schools have only coarse (weak) priorities, e.g. for students that have a sibling at that school already, then the matching will result in ties that must be broken somehow, usually by a random lottery. Proximity based priority (often the lowest priority in a school's priority list) can then become a strict priority which can be ordinally ranked. Using random lotteries does not guarantee the most efficient matching; however, it is possible to obtain a Pareto efficient, stable matching in the set of student optimal stable matchings (SOSM) when using only weak priorities, by employing a polynomial-time algorithm \cite{erdilWhatsMatterTieBreaking2008}. The trade-off is that there may be multiple SOSM in the set, and in that case the mechanism is no longer DST. %trade off between coarse priorities and strategy-proofness.

An attractive feature of a mechanism is to accommodate coarse preferences from students who are indifferent over a group of schools. DA cannot accomplish this by itself, but it is possible to obtain efficient, stable matchings even with ties, or to obtain student-optimal, stable matchings \cite{erdilTwosidedMatchingIndifferences2017}, though this is once again at the cost losing the DST property \cite{rothEconomicsMatchingStability1982}. An alternative approach to enabling coarse preferences is the incorporation of bundle allocation into the mechanism \cite{huangBundledSchoolChoice2025}. This allows students to select exogenously prepared bundles of schools with a single preference slot, while also providing any student the possibility to select any school within that bundle individually. Bundle allocation of this sort requires a second stage to allocate students who were matched with a bundle in the first stage to an individual school within that bundle, and additionally requires that all schools within a bundle share the same over-subscription criteria.

\subsection{Successive and sequential allocations in school choice}
In New York, there is evidence to suggest that attending a high quality middle school causes students to apply to high-quality high schools and that the effect is more powerful than information intervention alone \cite{anderssonSequentialSchoolChoice2024}. Furthermore, a dynamic model inspired by the empirical results attempts to characterise the value of attending a high-quality middle school on future allocation to high-quality high schools. It finds that high-quality middle schools induce a greater application rate to higher-quality high schools, and that this effect is much greater than the effect of high-school priorities for the purposes of admission. The authors then analyse the effects of removing geographical dependance on the priorities of middle and/or high schools, and find that the high school reform alone reduced income disparities in high schools, but the middle school reform reduced income disparities in middle and high schools, and when both reforms are implemented, the effects are reinforced in high schools.

\subsection{Socio-economic segregation in schools and residential areas}
Empirical evidence suggests that school choice has increased school segregation (the differences between schools) in many cities and contexts \cite{wilsonSchoolChoiceCity2019}. Particularly in England, the geographical constraints of the system with respect to the proximity-based over-subscription criteria restrict the feasible choices of disadvantaged families \cite{burgessParentalChoicePrimary2011a,burgessSchoolChoiceEngland2019}. This restriction in choice has the effect of re-introducing the geographical inequities that the school choice system is supposed to ameliorate, and may additionally cause residential sorting. 

Models that assume that school choice reduces the gap in quality between schools in a given town appear to incentivise the flight of the most and least advantaged parents to nearby towns that do not offer school choice, that is to say, advantaged parents will move to towns where their advantage can afford them better school quality, and disadvantaged parents will be displaced by residential price increases \cite{averyDistributionalConsequencesPublic2021}. Recent, model-based evidence seems to show there is a trade-off associated with proximity-based assignment (as compared with random assignment); proximity assignment seems to increase school segregation but reduce residential segregation \cite{greavesSchoolChoiceNeighborhood2024}. With proximity based school assignment, advantaged parents can self-select into areas with "better" schools, inducing an upward pressure on house value (and rent) in that area, the externalities of which seem to spill over onto non-parents; the same model shows an unambiguous loss in non-parents' utility under proximity-based over-subscription compared with random assignment.

Importantly, there is an implicit assumption that parents (as consumers) will rank higher performing schools higher in their preference lists, but some evidence seems to suggest that disadvantaged parents' choices are restricted by proximity-based over-subscription priorities \cite{burgessWhatParentsWant2015}. The same paper finds significant heterogeneity in preferences regarding school quality across socio-economic groups. Before the 2008 ban on IA mechanisms in England, advantaged parents would strategically ration their top ranked preference so as not to waste it on a school they may not be able to attain, which left room for disadvantaged but ambitious parents to obtain a few seats, which has been coined the \textit{competition-for-top-schools} effect, but the shift away from IA mechanisms has eliminated some risk from applying for performant schools, making them far more competitive and likely to be oversubscribed \cite{pathakSchoolAdmissionsReform2013}. Selective schools, like grammar schools in England and Wales, are particularly prone to this effect.

\subsection{Effects of segregation on urban environments}
There is evidence from Brazil that suggests that more integrated cities align people across the socio-economic spectrum by sharing spatial externalities, undermining private goods provisions and thus forming coalitions across class barriers to demand improvements in public goods \cite{xuSegregationSpatialExternalities2024}. As discussed above, school allocation outcomes could also be seen as spatial externalities.

Another study analyses data from the Moving to Opportunity (MTO) experiment conducted in the United States, whose main finding states that exposure of disadvantaged children to low-poverty environments in very early childhood had a significant effect on their long term earnings and wellbeing, with some evidence that it could disrupt patterns of generational poverty too \cite{chettyEffectsExposureBetter2016}. The same author also published another paper expanding on the exposure effect with a much larger dataset \cite{chettyImpactsNeighborhoodsIntergenerational2018}.

Segregated school environments seem to contribute to higher levels of youth criminal activity, especially in more disadvantaged neighbourhoods, leading to more total crime in the urban area \cite{billingsPartnersCrime2019}.

\subsection{Effects of transportation access on school allocation}

\subsection{Interventions on fixed mechanisms}
Evidence from agent-based models shows that by removing travel barriers between schools and communities, it is possible to reduce school segregation in an urban setting, based on Amsterdam. The model reduces travel time for communities to schools, and gets positive results without changing community homophily characteristics \cite{michailidisTacklingSchoolSegregation2024}. Another model uses reinforcement learning to edit graphs under a budget by adding a limited number of edges, and find that they increase utility and reduce inequity in a network modelled on Chicago, however they do not explicitly include the allocation mechanisms in the analysis, rather editing transport routes for existing allocations \cite{ramachandranGAEAGraphAugmentation2021}.

\section{Methodology}
\subsection{Research Objectives}
We aim to achieve three goals with this research:

\subsubsection{Design a new mechanism}
The mechanism must be diversity aware by default, and must not exacerbate existing socio-economic segregation, actively reducing it if possible. This could be achieved through changes to the priority structure to remove explicit ranking by proximity, coupled with a bundle allocation modification which incorporates a public transportation offering to particular bundles. Removing proximity dependence and replacing it with a rule that prioritises journey time (or removing a proximity/time priority altogether, since a student-optimal, stable, and constrained-efficient matching can be found using stable improvement cycles) while providing transport options for students will make participation in the mechanism more equitable. Using only coarse preferences and/or priorities could still lead to a potentially efficient or student optimal matching at the cost of the DST property, however the current constrained-preference system already loses this property and induces strategic behaviour; an analysis of manipulability of potential alternative mechanisms will provide a better yardstick by which to compare the mechanisms.

\subsubsection{Intervene in a fixed system}
Given a fixed allocation mechanism, we would like to design an intervention that can reduce segregation between schools. This would be a budget allocation problem where we would extend existing work to consider the costs of intervention and incorporate factors like heterogeneous agent preferences, more than two agent types, accurate edge distances, and nodes' internal spatial distributions. It will be important to analyse possible residential resorting effects in the intervention, since the literature suggests this is one of the key factors involved in school segregation with proximity-based over-subscription criteria. 

\subsubsection{Model the effects of sequential allocations}
It will be essential to understand how residential, school, and transport equilibria will evolve, and so it is important to model the effects that the designed mechanism will have if implemented. Beyond this, many secondary schools have priorities which benefit children who come from "feeder schools," and as seen above, middle schools appear to have a large impact on the application process for high schools in New York. Seeing how successive allocations and residential sorting are affected by transport or mechanism interventions will be important, not least because residence is often an endogenous factor in parental preferences.


\section{Impact}
\subsection{Sustainable Development Goals}
Major goals: 4 (Quality Education), and 11 (Sustainable Cities and Education).
Minor goals: 1 (No Poverty), and 13 (Climate Action).

For the research process itself, any models will require real data to calibrate them. This should be ethically sourced and responsibly managed with a plan set out before the acquisition. Any data collection that includes interviews must conform to the University's ethics framework, with appropriate documentation and communication with the project supervisors.

Regular supervisory meetings will be held to monitor progress and to assess both feasibility and ethics. Throughout the project there will be participation in both conferences and workshops with the intention to present work and incorporate feedback, while also introducing potential new stakeholders and collaborators. Outreach activities will connect the project to stakeholders outside the academic sphere.

The ultimate impact of the project has the goal of producing methods to reduce school segregation. The effects of the phenomenon have been briefly covered above, but continuous research will be done to keep pace with evolving perspectives in the literature. Combining the primary goal with that of increasing transport utility may incentivise the development and improvement of transport infrastructure, which is a significant public good in and of itself. The potential resorting effects of school choice must be carefully monitored, since they can be unpredictable and possibly detrimental to both the project outcomes and general urban outcomes. The effects of school choice on the quality of schools is still an active discussion, and should be carefully monitored to avoid inducing negative externalities.

Community outreach for the purposes of understanding local externalities may help to guide the research, augmenting it with real peoples' perspectives. Decisions are ultimately made by the author, but are subject to consultation by the supervisory team. If the research outcomes are found to be societally beneficial, it will be prudent to involve policymakers into the discussion, for an exchange of advice and experience.

%\subsection{Changes}
%Introducing changes to priorities and presenting schools alongside transport options (discounts?) as bundles to influence preferences may reduce dependence on private vehicle use, producing voting coalitions which demand improvements to public goods, leading to social and environmental sustainability.

\section{Plan of future work}
Work will follow a 6-month cycle of literature review, write-up, workshops, improvement then publishing or presenting to journals or conferences. The produced work will follow the research objectives stated above, while maintaining a continuous documentation of completed work for posterity.

\end{multicols}

% Bibliography
\break
\printbibliography

\end{document}
