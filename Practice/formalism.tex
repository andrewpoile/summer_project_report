\documentclass{article}
\usepackage{graphicx} % Required for inserting images

% Layout packages
\usepackage[margin=0.5in]{geometry}
\usepackage{multicol}

% Formatting packages
\usepackage{parskip}
\usepackage{csquotes}

% Body of document
\title{Sustainable Resource Allocation in Urban School Settings (STRAUSS)}
%\author{Andrew Poile}
%\date{August 2025}

\begin{document}

\maketitle

\section{Objects}
\begin{itemize}
    \item 3 disjoint sets; students, colleges, and routes:
    \subitem $S = \{s_1,...,s_{n_S}\}, C = \{c_1,...,c_{n_C}\}, R = \{r_1,...,r_{n_R}\}$.
    \item Routes can have a capacity of more than 1: $q_r \geq 1$.
    \item Each route serves one college. Routes are effectively taxi services.
    \item Routes are only available in selected areas (only feasible for a subset of students $S_{r_i} \in S$ for a route i).
    \item Students can select either a college, or (route, college) tuple, hereafter referred to as object "a" eg. for $s_1: a_{11} = (r_1,c_1), a_{23} = (r_2,c_3), etc$. Each object $a$ contains exactly one college.
    \item Student preferences over colleges and (route, college) tuples (that are available to them) are strict and complete.
    \item All valid "a"s have their corresponding (route, student) tuples added to the top priority bracket in the priority list of the college in "a", hereafter referred to as "b" eg. for $c_1: b_{11} = (r_1,s_1), ...$, for $c_3: b_{21} = (r_2,s_1), ...$, etc.
    \item 2 college priority brackets given by:
    \subitem Students \textit{within} a certain distance of a school selecting that school \textbf{without} transport AND students \textit{within} a selected area served by a route, selecting a school \textbf{with} transport are placed in the same priority bracket; all other students placed in the bracket below this.
    \item Ties broken randomly within priority brackets.
\end{itemize}

\section{Algorithm (modified deferred acceptance)}
\begin{enumerate}
\item All objects unassigned. $\mu = \emptyset$
\item While some student $s_n$ is unmatched:
    \begin{enumerate}
        \item $s_n$ proposes to the first preference on their preference list (either $c_j$ or $a_{ij} = (r_i,c_j)$):
        \item If applying to a college $c_j$ with no route then $\mu = \mu \cup \{s_n, c_j\}$. If applying to a college with a route $a_{ij}$ then $\mu = \mu \cup \{s_n, a_{ij}\}$. If the college is oversubscribed in either case:
        \begin{enumerate}
            \item The lowest priority object (either student $s_k$ or student-route tuple $b_{il} = (r_i,s_l)$) in the priority list for $c_j$ is identified:
            \begin{itemize}
                \item in the case of $s_k$, it is directly released from the matching: $\mu = \mu \backslash \{s_k, c_j\}$
                \item in the case of $b_{il}$ the complementary object $a_{ij} = (r_i,c_j)$ and student $s_l$ are released from the matching: $\mu = \mu \backslash \{s_l, a_{ij}\}$. If $b_{il}$ is released from $c_j$ then $s_l$ is also released from the route $r_i$.
            \end{itemize}
        \end{enumerate}
        \item If applying to a college with a route $a_{ij}$ then $\mu = \mu \cup \{s_n, a_{ij}\}$. If route $r_i$ has reached capacity:
        \begin{enumerate}
            \item The student $s_k$ from the lowest priority, complementary route-student tuple $b_{ik} = (r_i,s_k)$ competing for the route $r_i$ in the priority list for $c_j$ is released from the matching: $\mu = \mu \backslash \{s_k, a_{ij}\}$.
            \item This is valid whether the college is at capacity or not since the end result is a student being rejected from the college.
        \end{enumerate}
    \end{enumerate}
    
        %\subsubitem - All students lower in priority than $s_k$ are removed from the priority list of the school in $a_j$ (and all of those students remove objects that contain the school in $a_j$ from their preference lists).
\end{enumerate}

\section{Checking stability}
The algorithm outputs two objects: the (stable) matching and the list of unassigned students; both must be checked for blocking pairs.
\begin{enumerate}
    \item For the matching:
    \begin{enumerate}
        \item For each student, s, check their top preference:
        \begin{itemize}
            \item IF they are assigned their top preference then they cannot form a blocking pair.
            \item ELSE IF they are not assigned their top preference, begin checking the preferences starting with the most preferred:
            \begin{itemize}
                \item IF the preference has no route associated with it AND the associated college has spare capacity: forms a blocking pair.
                \item ELSE IF the preference has no route associated with it AND the associated college IS at capacity BUT the college prefers s to any of its assigned students: form a blocking pair.
                \item ELSE IF the preference does have a route associated with it, which is NOT at capacity AND the college has spare capacity: form a blocking pair.
                \item ELSE IF the preference does have a route associated with it, which is NOT at capacity AND the college IS at capacity BUT the college prefers s over any of its assigned students: form a blocking pair.
                \item ELSE IF the preference does have a route associated with it which IS at capacity AND the college is at capacity OR has spare capacity AND s is preferred to any routed student: form a blocking pair.
            \end{itemize}
        \end{itemize}
    \end{enumerate}
    \item For the list of unassigned students (if it exists):
    \begin{itemize}
        \item IF the preference has no route associated with it AND the associated college has spare capacity: forms a blocking pair.
        \item ELSE IF the preference has no route associated with it AND the associated college IS at capacity BUT the college prefers s to any of its assigned students: form a blocking pair.
        \item ELSE IF the preference does have a route associated with it, which is NOT at capacity AND the college has spare capacity: form a blocking pair.
        \item ELSE IF the preference does have a route associated with it, which is NOT at capacity AND the college IS at capacity BUT the college prefers s over any of its assigned students: form a blocking pair.
        \item ELSE IF the preference does have a route associated with it which IS at capacity AND the college is at capacity OR has spare capacity AND s is preferred to any routed student: form a blocking pair.
    \end{itemize}
\end{enumerate}

\section{Proof of stability}
Suppose some student $s$ and college $c$ OR tuple $(r,c)$ are unmatched in $\mu$ after running the algorithm, but $s$ prefers $c$/$(r,c)$ to $\mu(s)$. $c$/$(r,c)$ must be acceptable to $s$ since it is on their preference list, and so $s$ must have applied to $c$/$(r,c)$ before applying to $\mu(s)$ OR before exhausting their preference list. Since $s$ is not matched to $c$/$(r,c)$ when the algorithm terminates:
\begin{itemize}
    \item IF $s$ is NOT eligible for $r$; either $s$ applied to $c$, was accepted and was later rejected by $c$ in favour of a higher priority student/routed student, OR $c$ was already at capacity at the point of application and was rejected as the lowest priority student/routed student.
    \item ELSE IF $s$ IS eligible for $r$ and ONLY applied to $(r,c)$; either $s$ applied to $(r,c)$, was accepted and was later rejected from $r$ (and therefore also $c$) in favour of a higher priority routed student, OR $r$ was already at capacity at the point of application and $s$ was rejected as the lowest priority routed student.
    \item ELSE IF $s$ IS eligible for $r$ and applied to BOTH $(r,c)$ AND $c$; BOTH of [either $s$ applied to $(r,c)$, was accepted and was later rejected from $r$ (and therefore also $c$) in favour of a higher priority routed student, OR $r$ was already at capacity at the point of application and $s$ was rejected as the lowest priority routed student] AND [either $s$ applied to $c$, was accepted and was later rejected by $c$ in favour of a higher priority student/routed student, OR $c$ was already at capacity at the point of application and was rejected as the lowest priority student/routed student].
\end{itemize}

\end{document}